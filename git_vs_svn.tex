\documentclass[11pt]{article}
%Gummi|065|=)
\title{\textbf{Git vs. Subverion \\ -- \\ Der Vergleich }}
\author{Sven Eiermann}
\date{}
\begin{document}

\maketitle

\section{Vorteile beider Systeme}

Das sind die jeweiligen Vorteile der beiden Systeme:

Git sollten Sie immer dann bevorzugen, wenn Sie:

\begin{itemize}

\item    nicht auf eine dauerhafte Netzwerkverbindung angewiesen sein wollen, um überall an Ihrem Projekt arbeiten zu können.
  \item  im Falle eines Ausfalls oder Verlusts des Haupt-Repositorys abgesichert sein wollen.
    \item keinerlei Lese- und Schreibberechtigung für spezielle Verzeichnisse benötigen (wobei diese auf komplexem Weg auch mit Git eingerichtet werden können).
   \item  Wert auf eine sehr schnelle Übertragung der Änderungen legen.
\end{itemize}


Subversion ist die bessere Wahl, wenn Sie:


\begin{itemize}
    \item pfadbasierte Zugangsberechtigungen für verschiedene Bereiche Ihres Projektes benötigen.
    \item Ihre gesamte Arbeit an einem zentralen Ort bündeln möchten.
    \item mit vielen großen Binär-Dateien arbeiten.
    \item auch die Strukturen leerer Verzeichnisse vollständig aufzeichnen möchten (Git verwirft diese, da sie keinerlei Inhalt besitzen).
\end{itemize}

\end{document}
